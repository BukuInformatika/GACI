Codeigniter adalah sebuah framework untuk web yang dibuat dalam format PHP. Format yang dibuat ini selanjutnya dapat digunakan untu membuat sistem aplikasi web yang kompleks. Codeigniter dapat mempercepat proses pembuatan web, karena semua class dan modul yang dibutuhkan sudah ada dan programmer hanya tinggal menggunakannya kembali pada aplikasi web yang akan dibuat \cite{prabowo2015website}.

\section{Tutorial Install CodeIgniter 3}
    \begin{enumerate}
        \item Pertama download Framework CodeIgniter di \textbf{\textit{https://www.codeigniter.com/}}
        
	    \item Setelah mengunduh file CodeIgniter 3, ekstrak file tersebut menggunakan WinRAR atau 7Zip kedalam folder \textit{C:/xampp/htdocs} jika kamu menggunakan XAMPP atau \textit{/var/www/html}. jika kamu menggunakan Apache2 Standalone, setelah itu ubahlan nama foldernya menjadi namaapplikasi.
	    
	    \item Sekarang silahkan Kamu coba akses URL \textbf{\textit{http://localhost/ namaaplikasi/}} melalui browser Kamu, akan langsung ditampilkan halaman awal Codeigniter yang berarti Instalasi telah berhasil.
		\begin{figure}[!htbp]
    		\centering
    		\includegraphics[width=0.5\textwidth]{figures/CodeIgniter1.PNG}
    		\label{CodeIgniter1}
		\end{figure}
    \end{enumerate}
    
\section{Struktur CodeIginter}
\begin{enumerate}
    \item Folder \textbf{Application}, merupakan folder yang pada dasarnya menyimpan aplikasi yang sedang kita buat
    \item Folder \textbf{Cache}, merupakan folder yang menyimpan semua cache yang dibuat oleh cache library
    \item Folder \textbf{Config}, merupakan folder yang menyimpan informasi mengenai konfigurasi aplikasi seperti autoload, database, routes dan lainnya.
    \item Folder \textbf{Controller}, merupakan folder menyimpan controller - controller aplikasi yang dapat digunakan untuk menyusun aktivitas program .
    \item Folder \textbf{Core}, adalah folder untuk memperluas class class inti codeigniter.
    \item Folder \textbf{Helpers}, merupakan folder untuk menyimpan helpers.
    \item Folder \textbf{Hooks}, merupakan folder untuk menyimpan hooks untuk mengubah alur fungsi dari core Codeigniter
    \item Folder \textbf{Language}, merupakan folder untuk menyimpan bahasa - bahasa yang akan digunakan.
    \item Folder \textbf{Libraries}, merupakan folder untuk menyimpan library.
    \item Folder \textbf{Logs}, merupakan folder untuk menyimpan semua error log apabila error log diaktifkan.
    \item Folder \textbf{Models}, merupakan folder untuk menyimpan models yang akan mendefinisikan tabel dari database yang dapat kita gunakan oleh Controller yang kita buat untuk mengakses database.
    \item Folder \verb|third_party|, merupakan folder untuk menyimpan fungsi fungsi tambahan dalam cara kerja codeigniter.
    \item Folder \textbf{Views}, merupakan folder untuk menyimpan tampilan dari aplikasi yang kita buat.
    \item Folder \textbf{System}, merupakan folder untuk menyimpan sistem inti dari Codeigniter.
\end{enumerate}

\section{Konfigurasi CodeIgniter 3}
Di dalam folder config pada CodeIgniter terdapat berbagai macam file konfigurasi yang dapat kita atur sendiri nantinya. File tersebut dapat ditemukan pada folder \textit{C:/xampp/htdocs/algoritmagenetika/application/config/}.
\begin{enumerate}
    \item \textbf{autoload.php}, digunakan untuk menambahkan package, libraries, drivers, helper, atau custom config lainnya agar secara otomatis diload oleh codeigniter.
    \item \textbf{config.php}, digunakan untuk membuat pengaturan dasar untuk web app codeigniter anda, seperti \verb|base_url|, index page, cookie, proxy dan lain lain.
    \item \textbf{constants.php}, digunakan untuk kita dapat membuat constant baru.
    \item \textbf{database.php}, digunakan untuk mengatur koneksi web app kita ke database.
    \item \textbf{doctypes.php}, sebagai tempat penyimpanan deklarasi dokumen Doctype.
    \item \verb|foreign_chars.php|, sebagai tempat penyimpanan karakter karakter asing.
    \item \textbf{hooks.php}, digunakan untuk mendefine "hooks" untuk meng extends CI
    \item \textbf{memcached.php}, config yang memungkinkan kita mencache database, driver dan lain lain sehingga lebih efektif.
    \item \textbf{migration.php}, config yang memungkinkan kita melakukan database migration. Secara default dijadikan False.
    \item \textbf{mimes.php}, menyimpan array yang berisi tipe file untuk fungsi upload.
    \item \textbf{profiler.php}, digunakan untuk mengatur profiler yang berguna pada saat debugging.
    \item \textbf{routes.php}, digunakan untuk mengatur default controllerdan overide 404
    \item \textbf{smileys.php}, menyimpan array yang berisi smiley yang membantu helper emoticon.
    \item \verb|user_agents.php|, menyimpan data user agent, yang membantu class User Agen untuk mengidentifikasi browser, platform, robotdan datamobile device
\end{enumerate}

Pada konfigurasi yang saya lakukan hanya melukakan konfigurasi pada file autoload.php, config.php, database.php dan routes.php. Berikut cara konfigurasinya:
\begin{enumerate}
    \item Autoload.php
		\par Pada file ini saya meng-input libraries untuk support framework CodeIgniter ini terhadap database, \verb|form_validation| yang akan dibuat nantinya, pagination dan Session untuk mengaktifkan session pada CodeIgniter. Pada variable autoload helper saya meng-input url dan form semua di inputkan sesuai dengan kebutuhan pembuat aplikasi. Array tersebut akan di eksekusi secara automatis oleh CodeIgniter.
\begin{lstlisting}
$autoload['libraries'] = array('database','session','form_validation','pagination');
$autoload['helper'] = array('url','form','html');
\end{lstlisting}

    \item Config.php
        \par Pada config.php inputkan url utama aplikasi pada variable config \verb|base_url| seperti pada codingan dibawah:
\begin{lstlisting}
$config['base_url']	= 'http://localhost/algoritma_genetika/';
\end{lstlisting}
        
    \item Database.php
		\par Pada database.php konfigurasi yang dilakukan untuk mengkoneksikan database yaitu MySQL dengan aplikasi web berbasis framework CodeIgniter. Dapat dilihat pada line 72, hostname yang diiniputkan sesuai dengan hostname yang dipakai, disini saya menginputkna localhost dengan username default yaitu root password dikosongkan karena pada Xampp saya tidak menggunakan password. Pada database line 75 inputkan nama database sesuai dengan nama database yang ada pada MySQL.
\begin{lstlisting}
$active_group = 'default';
$query_builder = TRUE;
$db['default'] = array(
	'dsn'	=> '',
	'hostname' => 'localhost',
	'username' => 'root',
	'password' => '',
	'database' => 'produk_ga_fs',
	'dbdriver' => 'mysqli',
	'dbprefix' => '',
	'pconnect' => FALSE,
	'db_debug' => (ENVIRONMENT !== 'production'),
	'cache_on' => FALSE,
	'cachedir' => '',
	'char_set' => 'utf8',
	'dbcollat' => 'utf8_general_ci',
	'swap_pre' => '',
	'encrypt' => FALSE,
	'compress' => FALSE,
	'stricton' => FALSE,
	'failover' => array(),
	'save_queries' => TRUE
);
\end{lstlisting}
		
	\item Routes.php
		\par Pada file ini dilakukan konfigurasi dimana controller mana yang akan pertama di eksekusi ketika url dijalankan.
\begin{lstlisting}
$route['default_controller'] = "CTRL_Dashboard";
$route['404_override'] = '';
\end{lstlisting}
\end{enumerate}

\section{Konfigurasi Bootstrap dan Template CodeIgniter 3}
Ada berbagai macam konfigurasi bootstrap dan template terhadap CodeIgniter, baik secara install maupun dengan cara konfigurasi sendiri. Pada tutorial kali ini saya ingin menerapkan bootstrap dan template di CodeIgniter dengan cara cepat. Untuk yang ingin menggunakan cara instan, bisa dengan cara mengunjungi website \textit{w3layout.com} dan website yang menyediakan assets template dan bootstrap siap pakai. Berikut cara konfigurasi template dan bootstrap pada CodeIgniter:
\begin{enumerate}
    \item Siapkan template bootstrap yang sudah didownload
    \item Extrak file tersebut jika dalam bentuk .rar atau .zip
    \item Buat folder baru dengan nama assets terhadap aplikasi yang ingin di konfigurasi kemudian copy file hasil extrak tadi ke dalam folder tersebut.
		\begin{figure}[!htbp]
    		\centering
    		\includegraphics[width=0.5\textwidth]{figures/TBCI1.png}
    		\label{TBCI1}
		\end{figure}
		
	\item Setelah menkopi file kedalam folder assets, langka selanjutnya adalah memanggil config tersebut. Jangan lupa untuk membuat header dan footer ketika membuat website guna untuk mempermudah apabila terjadi perubahan terhadap beberapa menu.
	\item Copy isi dari index.html yang ada dalam assets kemudian buat file di dalam follder \textit{C:/xampp/htdocs/algoritmagenetika/application/views/dasboard.php} dengan format .php dan pastekan dalam file tersebut.
	\item Kemudian pisahkan antara header dan footer aplikasi anda.
	\item Pertama lakukan konfigurasi terhadap header dengan cara memanggil link dan script yang sudah di copy di dalam assets, berikut contoh pemanggilanya:
\begin{lstlisting}
<!DOCTYPE html>
<html lang="en">

<head>
    <meta charset="utf-8">
    <meta http-equiv="X-UA-Compatible" content="IE=edge">
    <meta name="viewport" content="width=device-width, initial-scale=1">
    <meta name="description" content="">
    <meta name="author" content="">

    <title>Penjadwalan Expediting</title>

    <link href="<?php echo base_url('assets/vendor/bootstrap/css/bootstrap.min.css'); ?>" rel="stylesheet">
    <link href="<?php echo base_url('assets/vendor/metisMenu/metisMenu.min.css'); ?>" rel="stylesheet">
    <link href="<?php echo base_url('assets/dist/css/sb-admin-2.css'); ?>" rel="stylesheet">
    <link href="<?php echo base_url('assets/vendor/morrisjs/morris.css'); ?>" rel="stylesheet">
    <link href="<?php echo base_url('assets/vendor/font-awesome/css/font-awesome.min.css'); ?>" rel="stylesheet" type="text/css">
    <script src="<?php echo base_url('assets/vendor/jquery/jquery.min.js'); ?>"></script>

    <!-- Tabel Responsive -->
    <link href="<?php echo base_url('assets/vendor/datatables-plugins/dataTables.bootstrap.css'); ?>" rel="stylesheet">
    <link href="<?php echo base_url('assets/vendor/datatables-responsive/dataTables.responsive.css'); ?>" rel="stylesheet">

    <!-- DatePicker -->
    <link href="<?php echo base_url('assets/datepicker/datepicker.css'); ?>" rel="stylesheet">
    <script type="text/javascript" src="<?php echo base_url('assets/datepicker/datepicker.js'); ?>"></script>

    <style type="text/css">
         body .frmModalMsg {
            width: 740px;
            margin-left: -280px;
         }
   
         #line-chart {
            height:300px;
            width:800px;
            margin: 0px auto;
            margin-top: 1em;
         }
         .brand { font-family: georgia, serif; }
         .brand .first {
         color: #ccc;
            font-style: italic;
         }
         .brand .second {
            color: #fff;
            font-weight: bold;
         }
         
         #loading-div-background{
            display: none;
            position: fixed;
            top: 0;
            left: 0;
            background: #fff;
            width: 100%;
            height: 100%;
        }
        
        #loading-div{
            width: 300px;
            height: 150px;
            background-color: #fff;
            border: 5px solid #1468b3;
            text-align: center;
            color: #202020;
            position: absolute;
            left: 50%;
            top: 50%;
            margin-left: -150px;
            margin-top: -100px;
            -webkit-border-radius: 5px;
            -moz-border-radius: 5px;
            border-radius: 5px;
        }
      </style>

      <script type="text/javascript">
        $(document).ready(function (){
            $("#loading-div-background").css({ opacity: 0.5 });
         <?php if(isset($clear_text_box)) { ?>    
            $('input[type=text]').each(function() {
                $(this).val('');
            });
         <?php } ?>
        });
    
        function ShowProgressAnimation(){
            $("#loading-div-background").show();
        }
         
         function change_get(){     
            var semester_tipe = document.getElementById('semester_tipe');
            var tahun_akademik = document.getElementById('tahun_akademik');
            window.location.href = "<?php echo base_url().'web/pengampu/' ?>" + semester_tipe.options[semester_tipe.selectedIndex].value  + "/"   + tahun_akademik.options[tahun_akademik.selectedIndex].value;     
         }
         
         function change_dosen_tidak_bersedia() {
            var kode_dosen = document.getElementById('kode_dosen');         
            window.location.href = "<?php echo base_url().'web/waktu_tidak_bersedia/' ?>" + kode_dosen.options[kode_dosen.selectedIndex].value;     
         }
         
        function get_matakuliah() {        
            var semester_tipe = document.getElementById('semester_tipe');
            $.ajax({
               type:"POST",
               async   : false,
               cache   : false,
               url: "<?php echo base_url()?>web/option_matakuliah_ajax/" + semester_tipe.options[semester_tipe.selectedIndex].value,
               success: function(msg){
                  //alert(msg);
                  $('#option_matakuliah').html(msg);
               }
            });
            return false;        
        }

        function delete_row(link,kode) {
            var answer =  confirm('Anda yakin ingin menghapus data ini?');
            if(answer){
               $.ajax({
                  type:"POST",
                  async   : false,
                  cache   : false,
                  url: "<?php echo base_url()?>" + link + kode,
                  success: function(msg){
                     var tr  = $('#row_' + kode);
                     tr.css("background-color","#FF3700");
                     tr.fadeOut(400, function(){
                       tr.remove();
                     });                  
                  }
               });
            }
            return false;
        }
        
        $(function() {
                applyPagination();
                function applyPagination() {
                 $("#ajax_paging a").click(function() {             
                   var url = $(this).attr("href");
                   $.ajax({
                     type: "POST",
                     data: "ajax=1",
                     url: url, 
                     success: function(msg) {
                       $('#content_ajax').fadeOut(0,function(){
                           $('#content_ajax').html(msg);
                           $("#content_ajax").removeAttr("style");
                           applyPagination();                 
                       }).fadeIn(0);                       
                     }
                   });              
                   return false;
                 });
               }
             });
      </script>
</head>

<body>
    <div id="wrapper">
        <nav class="navbar navbar-default navbar-static-top" role="navigation" style="margin-bottom: 0">
            <div class="navbar-header">
                <button type="button" class="navbar-toggle" data-toggle="collapse" data-target=".navbar-collapse">
                    <span class="sr-only">Toggle navigation</span>
                    <span class="icon-bar"></span>
                    <span class="icon-bar"></span>
                    <span class="icon-bar"></span>
                </button>
                <a class="navbar-brand" href="<?php echo base_url();?>">Algoritma Genetika</a>
            </div>

            <div class="navbar-default sidebar" role="navigation">
                <div class="sidebar-nav navbar-collapse">
                    <ul class="nav" id="side-menu">
                        <li>
                            <a href="<?php echo base_url();?>"><i class="fa fa-dashboard fa-fw"></i> Dashboard</a>
                        </li>
                        <li>
                            <a href="#"><i class="fa fa-files-o fa-fw"></i> Data<span class="fa arrow"></span></a>
                            <ul class="nav nav-second-level">
                                <li>
                                    <?php echo anchor('Web/index_vendor', '<i class="fa fa-file-text fa-fw"></i> Vendor'); ?>
                                </li>
                                <li>
                                    <?php echo anchor('Web/index_barang', '<i class="fa fa-file-text fa-fw"></i> Barang'); ?>
                                </li>
                                <li>
                                    <?php echo anchor('Web/index_bulan_tahun', '<i class="fa fa-file-text fa-fw"></i> Bulan Dan Tahun'); ?>
                                </li>
                            </ul>
                        </li>
                        <li>
                            <?php echo anchor('web/index_jadwal', '<i class="fa fa-table fa-fw"></i> Jadwal'); ?>
                        </li>
                    </ul>
                </div>
            </div>
        </nav>
\end{lstlisting}
		\par Lakukan pemanggilan terhadap semua code yang berbau href dan src dengan mengisikan kodingan seperti echo \verb|base_url/assets/link_yg_dituju|, terlihat seperti pada codingan diatas. Lakukan hal yang sama terhadap file footer.php. Setelah itu simpan.
		
	\item Selanjutnya membuat file untuk tampilan awal yaitu dashboard.php
	\item Edit codingan yang sudah dipastekan pada file dashboard.php yang ada pada folder \textit{C:/xampp/htdocs/algoritmagenetika/application/views/} sesuai tampilan yang diinginkan. Apabila memisahkan file header dan footer jangan lupa untuk memanggil file tersebut. Berikut codingan full dari dasboard.php:
\begin{lstlisting}
<?php $this->load->view('page/header') ?>

<div id="page-wrapper">
    <div class="row">
        <div class="col-lg-12">
            <h1 class="page-header">Dashboard</h1>
        </div>
    </div>
    <div class="row">
        <div class="col-lg-3 col-md-6">
            <div class="panel panel-primary">
                <div class="panel-heading">
                    <div class="row">
                        <div class="col-xs-3">
                            <i class="fa fa-tasks fa-5x"></i>
                        </div>
                        <div class="col-xs-9 text-right">
                            <div class="huge">Data</div>
                            <div>Vendor</div>
                        </div>
                    </div>
                </div>
                <a href="<?php echo base_url()?>index.php/Web/index_vendor">
                    <div class="panel-footer">
                        <span class="pull-left">View Details</span>
                        <span class="pull-right"><i class="fa fa-arrow-circle-right"></i></span>
                        <div class="clearfix"></div>
                    </div>
                </a>
            </div>
        </div>
        <div class="col-lg-3 col-md-6">
            <div class="panel panel-green">
                <div class="panel-heading">
                    <div class="row">
                        <div class="col-xs-3">
                            <i class="fa fa-tasks fa-5x"></i>
                        </div>
                        <div class="col-xs-9 text-right">
                            <div class="huge">Data</div>
                            <div>Barang</div>
                        </div>
                    </div>
                </div>
                <a href="<?php echo base_url()?>index.php/Web/index_barang">
                    <div class="panel-footer">
                        <span class="pull-left">View Details</span>
                        <span class="pull-right"><i class="fa fa-arrow-circle-right"></i></span>
                        <div class="clearfix"></div>
                    </div>
                </a>
            </div>
        </div>
        <div class="col-lg-3 col-md-6">
            <div class="panel panel-yellow">
                <div class="panel-heading">
                    <div class="row">
                        <div class="col-xs-3">
                            <i class="fa fa-tasks fa-5x"></i>
                        </div>
                        <div class="col-xs-9 text-right">
                            <div class="huge">Data</div>
                            <div>Bulan Dan Tahun</div>
                        </div>
                    </div>
                </div>
                <a href="<?php echo base_url()?>index.php/Web/index_bulan_tahun">
                    <div class="panel-footer">
                        <span class="pull-left">View Details</span>
                        <span class="pull-right"><i class="fa fa-arrow-circle-right"></i></span>
                        <div class="clearfix"></div>
                    </div>
                </a>
            </div>
        </div>
        <div class="col-lg-3 col-md-6">
            <div class="panel panel-red">
                <div class="panel-heading">
                    <div class="row">
                        <div class="col-xs-3">
                            <i class="fa fa-table fa-5x"></i>
                        </div>
                        <div class="col-xs-9 text-right">
                            <div class="huge">Data</div>
                            <div>Jadwal</div>
                        </div>
                    </div>
                </div>
                <a href="<?php echo base_url()?>index.php/Web/index_jadwal">
                    <div class="panel-footer">
                        <span class="pull-left">View Details</span>
                        <span class="pull-right"><i class="fa fa-arrow-circle-right"></i></span>
                        <div class="clearfix"></div>
                    </div>
                </a>
            </div>
        </div>
    </div>

    <div class="row">
    </div>
</div>

<?php $this->load->view('page/footer') ?>
\end{lstlisting}
	    \par Load view berarti meload file yang ada di dalam folder view, dalam artian codingan diatas akan meload file header yang ada di folder \textit{application/views/page/header.php}, begitu juga dengan file \textit{footer.php}. Simpan semua konfigurasi kemudian  jalankan aplikasi algoritmagenetika tersebut, dengan cara ketik link \textit{http://localhost/algoritmagenetika/} di browser kesayangan anda.
	    
    \item Berikut hasil dari codingan dashboard.php.
		\begin{figure}[!htbp]
    		\centering
    		\caption{Tampilan Awal Template / Bootsrap CodeIgniter 3}
    		\includegraphics[width=0.5\textwidth]{figures/TBCI3.png}
    		\label{TBCI3}
		\end{figure}
\end{enumerate}

\section{MVC dan CRUD CodeIgniter 3}
\begin{enumerate}
    \item Konfigurasi Model
    \begin{enumerate}
        \item Buat table sesuai dengan kebutuhan aplikasi
        \item Kemudian buat file baru dengan nama \verb|MDL_Barang.php| dalam folder \textit{C:/xampp/htdocs/algoritmagenetika/application/models/}.
        \item Pada tahap ini saya akan membuat model untuk table barang, berikut codingan yang saya gunakan:
\begin{lstlisting}
<?php
class MDL_Barang extends CI_Model{
	function __construct(){
		parent::__construct();
	}
    
    public function get_barang(){
		$hasil=$this->db->get('barang');
		if($hasil->num_rows() > 0){
			return $hasil->result();
		}else{
			return false;
        }
    }

	public function insert_barang($barang_data)
	{
		$this->db->insert('barang',$barang_data);
	}

	public function find_barang($kode_barang)
	{
		$hasil = $this->db->where('kode_barang',$kode_barang)->limit(1)->get('barang');
		if($hasil->num_rows() > 0){
			return $hasil->row();
		}else{
			return array();
		}
	}

	public function update_barang($kode_barang, $barang_data)
	{
		$this->db->where('kode_barang',$kode_barang)
		->update('barang',$barang_data);	
	}
	
	public function delete_barang($kode_barang)
	{
		$this->db->where('kode_barang',$kode_barang)
		->delete('barang');
	}

	public function detail_barang($kode_barang)
	{
		$hasil = $this->db->where('kode_barang',$kode_barang)->limit(1)->get('barang');
		if($hasil->num_rows() > 0){
			return $hasil->result();
		}else{
			return array();
		}
	}
}
\end{lstlisting}
            
        \par Penjelasan:
        \begin{enumerate}
            \item function \verb|get_barang| berfungsi untuk meload semua data yang ada di database barang
            \item function \verb|insert_barang| berfungsi untuk melakukan execute insert data ke dalam table barang
            \item function \verb|find_barang| berfungsi untuk mencari kode barang yang akan di edit
            \item function \verb|update_barang| berfungsi untuk melakukan execute update terhadap table barang
            \item function \verb|delete_barang| berfungsi untuk melakukan execute delete data pada table barang
            \item function \verb|detail_barang| berfungsi untuk melihat data lengkap barang sesuai dengan id yang dipanggil.
            
            \par Function-function di atas merupakan function dasar untuk melakukan proses CRUD di model CodeIgniter 3.
        \end{enumerate}
    \end{enumerate}
    
    \item Konfigurasi Controller
    \begin{enumerate}
        \item Buat file baru dengan nama web pada folder \textit{C/xampp/htdocs/algoritmagenetika/application/controllers/}, kemudian buat beberapa function crud.
        \item Dalam file web.php, load semua model agar semua function disatukan dalam satu file, seperti pada codingan dibawah:
\begin{lstlisting}
<?php if (!defined('BASEPATH')) exit('No direct script access allowed');
class Web extends CI_Controller
{
	function __construct()
    {
        parent::__construct();
		$this->load->model(array('MDL_Vendor',
								 'MDL_Barang',
								 'MDL_Bulan_Tahun',
								 'MDL_Hari',
								 'MDL_Jam',
								 'MDL_Pengampu',
								 'MDL_Waktu_Tidak_Bersedia',
								 'MDL_Jadwal'));
		include_once("genetik.php");
		define('IS_TEST','FALSE');
    }
\end{lstlisting}
    		
    	\item Kemudian buat function pertama yaitu function \verb|index_barang|, untuk menampilakn seluruh data yang ada pada table barang.
\begin{lstlisting}
public function index_barang(){
    $data['barang'] = $this->MDL_Barang->get_barang();
    $this->load->view('web/barang/index_barang', $data);
}
\end{lstlisting}
    		
        \item Function kedua yaitu function \verb|add_barang|, untuk melakukan proses input data dalam bentuk variable sesuai dengan field yang ada pada table barang  yang kemudian variable tersebut akan di lempar ke model untuk di inputkan kedalam table.
\begin{lstlisting}
public function add_barang()
{
	$this->form_validation->set_rules('nama_barang','Nama barang','required');
	$this->form_validation->set_rules('kategori_barang','Kategori Barang','required');
	$this->form_validation->set_rules('tingkat_kebutuhan','Tingkat Kebutuhan','required');
	$this->form_validation->set_rules('jumlah_dalam_kategori','Jumlah Dalam Kategori','required');
	
	if($this->form_validation->run() == FALSE){
        $this->load->view('web/barang/form_add_barang');
	}else{
		$barang_data = array (
			'nama_barang'			=> set_value('nama_barang'),
			'kategori_barang'		=> set_value('kategori_barang'),
			'tingkat_kebutuhan'		=> set_value('tingkat_kebutuhan'),
			'jumlah_dalam_kategori'	=> set_value('jumlah_dalam_kategori')
		);
		$this->MDL_Barang->insert_barang($barang_data);
		$this->session->set_flashdata('notif','Data Berhasil Di Simpan');
		redirect('web/index_barang');	
	}
}
\end{lstlisting}
    		
    	\item Selanjutnya function \verb|edit_barang|, untuk melakukan proses edit data barang yang kemudian variable yang di tampung akan di execute pada model function \verb|update_barang|.
\begin{lstlisting}
public function edit_barang($kode_barang)
{
	$this->form_validation->set_rules('nama_barang','Nama barang','required');
	$this->form_validation->set_rules('kategori_barang','Kategori Barang','required');
	$this->form_validation->set_rules('tingkat_kebutuhan','Tingkat Kebutuhan','required');
	$this->form_validation->set_rules('jumlah_dalam_kategori','Jumlah Dalam Kategori','required');
	
	if($this->form_validation->run() == FALSE){
		$data['barang'] = $this->MDL_Barang->find_barang($kode_barang);
		$this->load->view('web/barang/form_edit_barang', $data);
	}else{
		$barang_data = array (
			'nama_barang'			=> set_value('nama_barang'),
			'kategori_barang'		=> set_value('kategori_barang'),
			'tingkat_kebutuhan'		=> set_value('tingkat_kebutuhan'),
			'jumlah_dalam_kategori'	=> set_value('jumlah_dalam_kategori')
		);
		$this->MDL_Barang->update_barang($kode_barang, $barang_data);
		redirect('web/index_barang');
	}	
}
\end{lstlisting}
    		
    	\item Function \verb|delete_barang|, untuk melakukan hapus data berdasarkan id yang di execute.
\begin{lstlisting}
public function delete_barang($kode_barang)
{
	$this->MDL_Barang->delete_barang($kode_barang);
	redirect('web/index_barang');	
}
\end{lstlisting}
    \end{enumerate}
    
    \item Konfigurasi Views
    \begin{enumerate}
        \item Buat file dengan nama \verb|index_barang.php| pada folder \textit{C/xampp/htdocs/algoritmagenetika/application/views/web/barang/} kemdian buat desain untuk menampilkan data dari table barang, berikut contoh codingan untuk vies data barang.
\begin{lstlisting}
<?php $this->load->view('page/header') ?>
    <div id="page-wrapper">
        <div class="row">
            <div class="col-lg-12">
                <h1 class="page-header">Barang</h1>
            </div>
        </div>

        <div class="row">
            <div class="col-lg-12">
                <ol class="breadcrumb">
                    <?=anchor('Web/add_barang','Tambah Data Barang',['class'=>'btn btn-primary btn-sm','style'=>'float:left;'])?>
                    <div style="clear: both;"></div>
                </ol>
            </div>
        </div>

        <div class="row">
            <div class="container table-responsive">
                <div class="col-lg-11">
                    <table id="dataTables-example" class="table table-hover">
                    <thead>
                        <tr>
                            <th class="header">Nama Barang</th>
                            <th class="header">Kategori Barang</th>
                            <th class="header">#</th>
                        </tr>
                    </thead>
                    <tbody>
                        <?php foreach($barang as $row) : ?> 
                        <tr>
                            <td><?=$row->nama_barang?></td>
                            <td><?=$row->kategori_barang?></td>
                            <td>
                                <center>
                                    <?=anchor('Web/detail_barang/' . $row->kode_barang,'Detail',['class'=>'btn btn-info btn-sm'])?>
                                    <?=anchor('Web/edit_barang/' . $row->kode_barang,'Ubah',['class'=>'btn btn-default btn-sm'])?>
                                    <?=anchor('Web/delete_barang/' . $row->kode_barang,'Hapus',['class'=>'btn btn-danger btn-sm','onclick'=>'return confirm(\'Apakah Anda Yakin ?\')'])?>
                                </center>
                            </td>
                        </tr>
                        <?php endforeach; ?>
                    </tbody>
                    </table>
                </div>
            </div>
        </div>
    </div>
<?php $this->load->view('page/footer') ?>
\end{lstlisting}
    		
    	\item Selanjutnya membuat file baru didalam folder yang sama, dengan nama file \verb|form_add_barang.php|, untuk view terhadap function tambah barang.
\begin{lstlisting}
<?php $this->load->view('page/header') ?>
    <div id="page-wrapper">
        <div class="row">
            <div class="col-lg-12">
                <h1 class="page-header">Tambah Data Barang</h1>
                <ol class="breadcrumb">
                  <li><?php echo anchor('web/index_barang', '<i class="fa fa-file-text fa-fw"></i> Data Barang'); ?></li>
                  <li class="active"><i class="fa fa-tasks fa-fw"></i> Tambah Data Barang</li>
                  
                  <div style="clear: both;"></div>
                </ol>
            </div>
        </div>

        <div class="row">
            <div class="col-lg-11">
                <?=form_open_multipart('web/add_barang/',['class'=>'form-horizontal'])?>

                    <?php $error = form_error("nama_barang", "<p class='text-danger'>", '</p>'); ?>
                    <div class="form-group <?php echo $error ? 'has-error' : '' ?>">
                        <label class="col-sm-2 control-label">Nama barang</label>
                        <div class="col-sm-10">
                            <input type="text" class="form-control" name="nama_barang" value="<?= set_value('nama_barang') ?>">
                        </div>
                    </div>
                    <?php echo $error; ?>

                    <?php $error = form_error("kategori_barang", "<p class='text-danger'>", '</p>'); ?>
                    <div class="form-group <?php echo $error ? 'has-error' : '' ?>">
                        <label class="col-sm-2 control-label">Kategori Barang</label>
                        <div class="col-sm-10">
                            <select class="form-control" name="kategori_barang">
                                <option value=''>--Pilih Kategori Barang--</option>
                                <option value='Piping'>Piping</option>
                                <option value='Mechanical'>Mechanical</option>
                                <option value='Instrument'>Instrument</option>
                                <option value='Electrical'>Electrical</option>
                                <option value='Safety'>Safety</option>
                            </select>
                        </div>
                    </div>
                    <?php echo $error; ?>

                    <?php $error = form_error("tingkat_kebutuhan", "<p class='text-danger'>", '</p>'); ?>
                    <div class="form-group <?php echo $error ? 'has-error' : '' ?>">
                        <label class="col-sm-2 control-label">Tingkat Kebutuhan</label>
                        <div class="col-sm-10">
                            <select class="form-control" name="tingkat_kebutuhan">
                                <option value=''>--Pilih Tingkat Kebutuhan--</option>
                                <option value='1'>1 (Dibutuhkan Paling Akhir)</option>
                                <option value='2'>2 (Dibutuhkan Setelah Point 3 Selesai)</option>
                                <option value='3'>3 (Dibutuhkan Setelah Barang Utama Datang)</option>
                                <option value='4'>4 (Dibutuhkan Paling Utama)</option>
                            </select>
                        </div>
                    </div>
                    <?php echo $error; ?>

                    <?php $error = form_error("jumlah_dalam_kategori", "<p class='text-danger'>", '</p>'); ?>
                    <div class="form-group <?php echo $error ? 'has-error' : '' ?>">
                        <label class="col-sm-2 control-label">Jumlah Dalam Kategori</label>
                        <div class="col-sm-10">
                            <select class="form-control" name="jumlah_dalam_kategori">
                                <option value=''>--Pilih Jumlah Dalam Kategori--</option>
                                <option value='1'>Kurang Dari 10 Barang</option>
                                <option value='2'>Kurang Dari 20 Barang</option>
                                <option value='3'>Kurang Dari 30 Barang</option>
                                <option value='4'>Kurang Dari 40 Barang</option>
                                <option value='5'>Kurang Dari 50 Barang</option>
                                <option value='6'>Kurang Dari 60 Barang</option>
                                <option value='7'>Kurang Dari 70 Barang</option>
                                <option value='8'>Kurang Dari 80 Barang</option>
                            </select>
                        </div>
                    </div>
                    <?php echo $error; ?>

                    <div class="form-group">
                        <div class="col-sm-offset-2 col-sm-10">
                            <button type="submit" class="btn btn-primary pull-right">Simpan</button>
                        </div>
                    </div>
               </form>
            </div>
        </div>
        
    </div>
<?php $this->load->view('page/footer') ?>

\end{lstlisting}
    		\par Jangan lupa untuk menutup form dan pastikan name pada codingan views sesuai dengan data variable yang akan di execute pada function \verb|add_barang| yang terdapat pada controller tadi.
    		
    	\item Masih dalam folder yang saya, buat file baru untuk views terhadap function edit data barang dengan nama \verb|form_edit_barang.php|. Pada edit barang, hal pertama yang harus dilakukan setelah menekan tombol edit barang adalah memanggil semua data yang ada pada table agar dapat diedit, berikut contoh pemanggilan data pada form edit data barang.
\begin{lstlisting}
<?php
$kode_barang    = $barang->kode_barang;
    if($this->input->post('is_submitted')){
        $nama_barang            = set_value('nama_barang');
        $kategori_barang        = set_value('kategori_barang');
        $tingkat_kebutuhan      = set_value('tingkat_kebutuhan');
        $jumlah_dalam_kategori  = set_value('jumlah_dalam_kategori');
    }else{
        $nama_barang            = $barang->nama_barang;
        $kategori_barang        = $barang->kategori_barang;
        $tingkat_kebutuhan      = $barang->tingkat_kebutuhan;
        $jumlah_dalam_kategori  = $barang->jumlah_dalam_kategori;
    }
?>
\end{lstlisting}
    		
    		\par Setelah memanggil variable yang akan diedit, dibawah codingan tersebut buat form untuk view function edit data barang.
\begin{lstlisting}
<?php $this->load->view('page/header') ?>
    <div id="page-wrapper">
        <div class="row">
            <div class="col-lg-12">
                <h1 class="page-header">Tambah Data barang</h1>
                <ol class="breadcrumb">
                  <li><?php echo anchor('web/index_barang', '<i class="fa fa-file-text fa-fw"></i> Data Barang'); ?></li>
                  <li class="active"><i class="fa fa-tasks fa-fw"></i> Edit Data Barang</li>
                  <div style="clear: both;"></div>
                </ol>
            </div>
        </div>

        <div class="row">
            <div class="col-lg-11">
                <?=form_open_multipart('web/edit_barang/' . $kode_barang,['class'=>'form-horizontal'])?>

                    <?php $error = form_error("nama_barang", "<p class='text-danger'>", '</p>'); ?>
                    <div class="form-group <?php echo $error ? 'has-error' : '' ?>">
                        <label class="col-sm-2 control-label">Nama barang</label>
                        <div class="col-sm-10">
                            <input type="text" class="form-control" name="nama_barang" value="<?= $nama_barang ?>">
                        </div>
                    </div>
                    <?php echo $error; ?>

                    <?php $error = form_error("kategori_barang", "<p class='text-danger'>", '</p>'); ?>
                    <div class="form-group <?php echo $error ? 'has-error' : '' ?>">
                        <label class="col-sm-2 control-label">Kategori barang</label>
                        <div class="col-sm-10">
                            <select class="form-control" name="kategori_barang">
                                <option value='<?= $kategori_barang ?>'><?= $kategori_barang ?></option>
                                <option value='Piping'>Piping</option>
                                <option value='Mechanical'>Mechanical</option>
                                <option value='Instrument'>Instrument</option>
                                <option value='Electrical'>Electrical</option>
                                <option value='Safety'>Safety</option>
                            </select>
                        </div>
                    </div>
                    <?php echo $error; ?>

                    <?php $error = form_error("tingkat_kebutuhan", "<p class='text-danger'>", '</p>'); ?>
                    <div class="form-group <?php echo $error ? 'has-error' : '' ?>">
                        <label class="col-sm-2 control-label">Tingkat Kebutuhan</label>
                        <div class="col-sm-10">
                            <select class="form-control" name="tingkat_kebutuhan">
                                <option value='<?= $tingkat_kebutuhan ?>'><?= $tingkat_kebutuhan ?></option>
                                <option value='1'>1 (Dibutuhkan Paling Akhir)</option>
                                <option value='2'>2 (Dibutuhkan Setelah Point 3 Selesai)</option>
                                <option value='3'>3 (Dibutuhkan Setelah Barang Utama Datang)</option>
                                <option value='4'>4 (Dibutuhkan Paling Utama)</option>
                            </select>
                        </div>
                    </div>
                    <?php echo $error; ?>

                    <?php $error = form_error("jumlah_dalam_kategori", "<p class='text-danger'>", '</p>'); ?>
                    <div class="form-group <?php echo $error ? 'has-error' : '' ?>">
                        <label class="col-sm-2 control-label">Jumlah Dalam Kategori</label>
                        <div class="col-sm-10">
                            <select class="form-control" name="jumlah_dalam_kategori">
                                <option value='<?= $jumlah_dalam_kategori ?>'><?= $jumlah_dalam_kategori ?></option>
                                <option value='1'>Kurang Dari 10 Barang</option>
                                <option value='2'>Kurang Dari 20 Barang</option>
                                <option value='3'>Kurang Dari 30 Barang</option>
                                <option value='4'>Kurang Dari 40 Barang</option>
                                <option value='5'>Kurang Dari 50 Barang</option>
                                <option value='6'>Kurang Dari 60 Barang</option>
                                <option value='7'>Kurang Dari 70 Barang</option>
                                <option value='8'>Kurang Dari 80 Barang</option>
                            </select>
                        </div>
                    </div>
                    <?php echo $error; ?>

                    <div class="form-group">
                        <div class="col-sm-offset-2 col-sm-10">
                            <button type="submit" class="btn btn-primary pull-right">Simpan</button>
                        </div>
                    </div>
               </form>
            </div>
        </div>
        
    </div>
<?php $this->load->view('page/footer') ?>
\end{lstlisting}
    		\par Pastikan value terisi, agar saat melakukan edit data, yang diedit dapat terlihat dan pastikan pada name sesuai dengan field yang ada pada table dan function \verb|edit_barang| yang ada pada folder controller.
    		
    	\item Simpan semua file tersebut kemudian jalankan aplikasi ini dengan load link \textit{http://localhost/algoritmagenetika/} pada browser kesayangan anda (jangan lupa untuk jalankan apache dan mysql xampp terlebih dahulu). Berikut tampilan CRUD dari codingan yang di buat.
    	\begin{enumerate}
    	    \item Views Barang
    		\begin{figure}[!htbp]
        		\centering
        		\caption{Tampilan Hasil MVC dan CRUD (Views Barang)}
        		\includegraphics[width=0.4\textwidth]{figures/Views5.png}
        		\label{Views5}
    		\end{figure}
    		
    		\item Views Tambah Barang
    		\begin{figure}[!htbp]
        		\centering
        		\caption{Tampilan Hasil MVC dan CRUD (Tambah Barang)}
        		\includegraphics[width=0.4\textwidth]{figures/Views6.png}
        		\label{Views6}
    		\end{figure}
    		
    		\item Views Edit Barang
    		\begin{figure}[!htbp]
        		\centering
        		\caption{Tampilan Hasil MVC dan CRUD (Edit Barang)}
        		\includegraphics[width=0.4\textwidth]{figures/Views7.png}
        		\label{Views7}
    		\end{figure}
    	\end{enumerate}
    \end{enumerate}
\end{enumerate}